\documentclass{article}[12 pt]
\usepackage{amssymb}
\usepackage{amsthm}
\usepackage{amsmath}
\usepackage{appendix}
\usepackage{array}
\usepackage{geometry}
\usepackage{enumitem}
\usepackage{graphicx}
\usepackage{subfig}
\usepackage{caption}
\usepackage{url}
\usepackage{float}
\usepackage{pdfpages}
\usepackage{shortvrb}
\usepackage{mathtools}
\usepackage{multirow}
\usepackage{hyperref}
\usepackage{commath}
\usepackage{tabularx}
\usepackage{bm}


\def\BibTeX{{\rm B\kern-.05em{\sc i\kern-.025em b}\kern-.08em
		T\kern-.1667em\lower.7ex\hbox{E}\kern-.125emX}}

\graphicspath{{"E:/University of Florida/Classes/2019_08_Principe_Deep_Learning/Homework/HW05/Report/Images/"}{"C:/Users/Conma/Documents/2019_08_Principe_Deep_Learning/Homework/HW05/Report/Images/"}{"/media/cmccurley/0000-0001/University of Florida/Classes/2019_08_Principe_Deep_Learning/Homework/HW05/Report/Images/"}}
\geometry{margin=1 in}

\newcommand{\smallvskip}{\vspace{5 pt}}
\newcommand{\medvskip}{\vspace{30 pt}}
\newcommand{\bigvskip}{\vspace{100 pt}}
\newcommand{\tR}{\mathtt{R}}




\begin{document}
	
\begin{center}
	\textbf{\Large Connor McCurley} \\
	EEE 6814 \qquad \textbf{\large Homework 5} \qquad Fall 2019 
\end{center}


This assignment tasked us with applying multi-resolution PCA as a pre-processing procedure for Fashion-MNIST classification. To be honest, I never fully understood the implementation of PCA-M demonstrated in the paper (the steps were really not clear). So I implemented my own version of multi-resolution PCA (which may or may not be valid).   To do this, I took the full-resolution 

%\begin{figure}[h]
%	\centering
%	\subfloat[][]{\includegraphics[width=.3\textwidth]{"Example_0_full_res"}}\quad
%	\subfloat[][]{\includegraphics[width=.3\textwidth]{"Example_0_multi_res"}}\\
%	\subfloat[][]{\includegraphics[width=.3\textwidth]{"Example_10000_full_res"}}\quad
%	\subfloat[][]{\includegraphics[width=.3\textwidth]{"Example_10000_multi_res"}}
%	\caption{Full resolution images (left) and corresponding multi-resolution PCA features (right) for two training examples.}
%	\label{fig:sub1}
%\end{figure}
%
%\begin{figure}[!h]
%	\centering
%	\subfloat[][]{\includegraphics[width=.3\textwidth]{"Example_20000_full_res"}}\quad
%	\subfloat[][]{\includegraphics[width=.3\textwidth]{"Example_20000_multi_res"}}\\
%	\subfloat[][]{\includegraphics[width=.3\textwidth]{"Example_30000_full_res"}}\quad
%	\subfloat[][]{\includegraphics[width=.3\textwidth]{"Example_30000_multi_res"}}
%	\caption{Full resolution images (left) and multi-resolution PCA features (right) for two training examples.}
%	\label{fig:sub2}
%\end{figure}
%
%\begin{figure}[!h]
%	\centering
%	\subfloat[][]{\includegraphics[width=.3\textwidth]{"Example_40000_full_res"}}\quad
%	\subfloat[][]{\includegraphics[width=.3\textwidth]{"Example_40000_multi_res"}}\\
%	\subfloat[][]{\includegraphics[width=.3\textwidth]{"Example_50000_full_res"}}\quad
%	\subfloat[][]{\includegraphics[width=.3\textwidth]{"Example_50000_multi_res"}}
%	\caption{Full resolution images (left) and multi-resolution PCA features (right) for two training examples.}
%	\label{fig:sub3}
%\end{figure}



%\begin{figure}[h]
%	\centering
%	\subfloat[][]{\includegraphics[width=.4\textwidth]{"pcam_no_fft_learning_curve"}}\quad
%	\subfloat[][]{\includegraphics[width=.4\textwidth]{"pcam_no_fft_confusion_mat"}}\\
%	\subfloat[][]{\includegraphics[width=.4\textwidth]{"pcam_fft_learning_curve"}}\quad
%	\subfloat[][]{\includegraphics[width=.4\textwidth]{"pcam_fft_confusion_mat"}}
%	\caption{Top: Learning curves and confusion matrix from the network trained with Multi-resolution PCA features.  Bottom: Learning curves and confusion matrix from the network trained with frequency-domain Multi-resolution PCA features.}
%	\label{fig:pcam_confusion_mats}
%\end{figure}
%
%\begin{figure}[h]
%	\centering
%	\subfloat[][]{\includegraphics[width=.4\textwidth]{"base_confusion_mat"}}\quad
%	\subfloat[][]{\includegraphics[width=.4\textwidth]{"pca_100_confusion_mat"}}\\
%	\caption{Left: Confusion matrix for the base model trained on raw images. Right: Confusion matrix for model trained with images projected onto the first 100 principal components of the training set.}
%	\label{fig:base_confusion_mats}
%\end{figure}








\end{document}
